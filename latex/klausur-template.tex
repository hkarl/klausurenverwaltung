\documentclass[addpoints,12pt]{exam}
% \documentclass[answers,addpoints,12pt]{exam}

\usepackage[ngerman]{babel}
\usepackage[utf8]{inputenc}
\usepackage{color}


% standard packages to make life easier:
\usepackage{tikz}
\usepackage{amsmath}
\usepackage[version=3]{mhchem}
\usepackage{chemfig}


\input{header}

\renewcommand{\solutiontitle}{\noindent\textbf{Lösung:}\enspace}
\shadedsolutions 
\definecolor{SolutionColor}{rgb}{0.8,0.9,1}

\pagestyle{head}
\runningheadrule
\firstpageheader{\klausurfach}{\klausurnummer}{\klausurdatum}
\runningheader{\klausurfach}
              {\klausurnummer, Seite \thepage\ von \numpages}
              {\klausurdatum}

\pointpoints{\,Punkt}{\,Punkte}
 

\hqword{Fragen: }
\hpword{Punkte: }
\hpgword{Seite: }
\htword{Gesamt }
\hsword{Erreicht: }



\begin{document}

\ifdefined\loesung
  \printanswers
\fi 

\begin{center}
\fbox{\fbox{\parbox{5.5in}{\centering
\boxtext}}}
\end{center}

\vspace{0.1in}
\makebox[\textwidth]{Name:\enspace\hrulefill}

\pointsinrightmargin
  

\input{questions}

 

% \ref\hrule
% \vspace{0.1cm}
% \noindent Der Test besteht aus \numquestions{} Fragen, bei denen insgesamt
% \numpoints{} Punkte erreicht werden können. 

 


% \begin{center}
%   \gradetable[h][questions]
% \end{center}

\end{document}
